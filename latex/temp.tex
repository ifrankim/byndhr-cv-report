\documentclass[10.5pt, a4paper]{report}
\usepackage{tgadventor}
\renewcommand*\familydefault{\sfdefault} %% Only if the base font of the document is to be sans serif
\usepackage[T1]{fontenc}
\usepackage{graphicx}
\usepackage{fancyhdr}
\usepackage{tabularx}
\usepackage{enumitem}
\usepackage{amssymb}
\usepackage[portuguese]{babel}
\usepackage[top=3.9cm, headheight=2cm, foot=2.2cm, left=1.7cm, right=1.7cm]{geometry} % Ajusta a margem superior e a altura do cabeçalho
\usepackage{lastpage} % Pacote para calcular o número total de páginas


% Configuração do cabeçalho e rodapé
\pagestyle{fancy}
\fancyhf{}
\rhead{\includegraphics[height=2.7cm]{./assets/logo.png}} % Logo no canto superior direito
\renewcommand{\headrulewidth}{0pt} % Remove a linha horizontal do cabeçalho
\rfoot{
  \fontsize{8}{9}\selectfont
  Alameda Santos, 415 – 10° andar \\
  01419-913 – Paraiso – São Paulo, SP, Brasil \\
  Página \textbf{\thepage}\ de \textbf{\pageref{LastPage}} % Mostra o número da página atual e o número total de páginas
}
% Personalizar títulos de seções
\usepackage{titlesec}
\titleformat{\section}
  {\normalfont\fontsize{13}{12}\bfseries}{}{0em}{}[\vspace{-0pt}]
\titleformat{\subsection}
  {\normalfont\large\bfseries}{}{0em}{}
\titleformat{\subsubsection}
  {\normalfont\normalsize\bfseries}{}{0em}{}

% Remover números de seção
\setcounter{secnumdepth}{0}

% Remover identação
\setlength{\parindent}{0pt}

\begin{document}
\rule{\textwidth}{0.1pt}\vspace{-10pt}

\section{RAZÃO SOCIAL DA VAGA}

\begin{minipage}[t]{0.3\textwidth}
  POSIÇÃO \\
  CANDIDATO
\end{minipage}%
\begin{minipage}[t]{0.4\textwidth}
  NOME DA POSIÇÃO \\
  FELIPE TAKEO KIKUCHI
\end{minipage}


\vspace{5pt}


\begin{enumerate}[label={}, leftmargin=0pt, topsep=8pt, itemsep=0pt]
  \item Itatiba / SP
  \item (11) 97465-3750
  \item Kikuchi.takeo@hotmail.com
  \item Brasileiro, 31 anos
\end{enumerate}


\vspace{-5pt}
\rule{\textwidth}{0.1pt}\vspace{-10pt}

\section{TRAJETÓRIA DE CARREIRA}

{\renewcommand{\arraystretch}{1.6}
\begin{tabular}{@{}p{0.68\textwidth}p{0.3\textwidth}@{}}
\textbf{Kromberg \& Schubert do Brasil} & 12/2017 - até o momento \\
            Analista de Qualidade Jr / Pl. & 12/2017 - até o momento \\
        Inspetor de Qualidade & 10/2010 - 10/2017 \\
        Auxiliar de Produção & 05/2010 - 09/2010 \\
        \end{tabular}
        }


\section{EXPERIÊNCIA PROFISSIONAL}

\vspace{10pt}
\section{KROMBERG \& SCHUBERT DO BRASIL}

Analista de Qualidade Jr / Pl.
\vspace{3pt}\\
{\footnotesize 12/2017 – até o momento}
\begin{enumerate}[leftmargin=32pt, topsep=15pt, itemsep=8pt, labelsep=10pt, align=left]
  \item Gerenciamento de atividades da Gestão de Qualidade de Fornecedores para todas as plantas da América Latina, com planejamento e atuando como auditor treinee em auditorias à fornecedores em acordo com as normas IATF 16949, VDA 6.3 e regulamentos da companhia.
  \item Suporte as plantas da América Latina com relação a resolução de não conformidades abertas junto aos fornecedores (visitas presenciais, reuniões periódicas, avaliação de plano de ação).
  \item Gestão de custos de não-qualidade e escalação de fornecedores para resolução de não conformidades.
\end{enumerate}

Inspetor de Qualidade
\vspace{3pt}\\
{\footnotesize 10/2010 – 10/2017}
\begin{enumerate}[leftmargin=32pt, topsep=15pt, itemsep=8pt, labelsep=10pt, align=left]
  \item Monitoramento e acompanhamento de processos produtivos de montagem e de crimpagem, no que diz respeito à análise de modos de falha e ações corretivas.
  \item Verificação e liberação 100\% em peças finais
  \item Leitura e interpretação de desenhos técnicos (produto e peça/componente)
  \item Gerenciamento, análise e tratativa de não conformidades de peças compradas de fornecedores externos (nacionais e importados)
  \item Análises de ações corretivas através de metodologia 8D Report
  \item Gerenciamento e follow up de custos de não qualidade relacionados a peças compradas de fornecedores externos
\end{enumerate}

Auxiliar de Produção
\vspace{3pt}\\
{\footnotesize 05/2010 – 09/2010}
\begin{enumerate}[leftmargin=32pt, topsep=15pt, itemsep=8pt, labelsep=10pt, align=left]
  \item Montagem e acabamento final de chicotes elétricos automotivo.
\end{enumerate}



\section{FORMAÇÃO ACADÊMICA}

\begin{enumerate}[label={\resizebox{0.4em}{!}{$\blacksquare$}}, leftmargin=32pt, topsep=15pt, itemsep=8pt, labelsep=10pt, align=left]\item Bacharel em Administração de Empresas
\newline Universidade São Francisco – Concluído em 2020
\item Ensino Médio
\newline E.E. Prof.ª Ivony de Camargo Sales – Concluído em 2009
\end{enumerate}

\section{CURSOS}
\begin{enumerate}[label={\resizebox{0.4em}{!}{$\blacksquare$}}, leftmargin=32pt, topsep=15pt, itemsep=0pt, labelsep=10pt, align=left]
  \item Inglês Intermediário – Escrita e Oral – Wizard Idiomas – conclusão em 2010
  \item FMEA AIAG-VDA 1ª Edição 2019 – IQA – 2023
  \item Certificação – Auditor Automotivo baseado na ISO 19011:2018 – Interaction Plexus – 2023
  \item VDA 6.5 - 3ª Edição - Auditoria de Produto – Interaction Plexus – 2022
  \item VDA 6.3:2016 – IQA – 2022
  \item MASP – IQA – 2022
  \item MSA 4º Edição – Interaction Plexus – 2022
  \item IATF16949 + ISO9001 – Interaction Plexus – 2022
  \item Requisitos Específicos Volkswagen – Interaction Plexus – 2022
  \item APQP e PPAP – Interaction Plexus – 2019
  \item VDA 2 – 5º Edição 2012 – Interaction Plexus – 2019
  \item Pacote Office – 
\end{enumerate}

\section{IDIOMAS}
\vspace{8pt}
\textbf{INGLÊS (4): OPERACIONAL}
\begin{enumerate}[leftmargin=32pt, topsep=16pt, itemsep=-2pt, align=left]
  \item Pré-Elementar
  \item Elementar
  \item Pré-Operacional
  \item \textbf{Operacional}
  \item Avançado
  \item Fluente
\end{enumerate}
\vspace{15pt}
\textbf{PRONÚNCIA}: É influenciada pela língua materna, mas apenas algumas vezes interfere na compreensão.
\newline\newline
\textbf{ESTRUTURA}: Aspectos gramaticais e padrões estruturais das frases são geralmente bem controlados. Erros podem acontecer em circunstâncias inesperadas.
\newline\newline
\textbf{VOCABULÁRIO}: A precisão do vocabulário é geralmente suficiente para comunicar-se em tópicos co- muns e relacionados ao trabalho.
\newline\newline
\textbf{FLUÊNCIA}: Capaz de se expressar verbalmente e se fazer compreendido, mas a construção das fra- ses e pausas são geralmente inapropriadas, mas não comprometem a compreensão do discurso.
\newline\newline
\textbf{COMPREENSÃO}: Na maioria das vezes, é precisa em tópicos comuns e relacionados ao trabalho quando o sotaque é suficientemente compreensível para uma comunidade internacional.
\newline\newline
\textbf{INTERAÇÕES}: As respostas costumam ser imediatas, apropriadas e informativas, e consegue iniciar e manter diálogos mesmo quando surgem eventos inesperados.
\vspace{25pt}
\newline\newline



\section{INFORMAÇÕES ADICIONAIS}
A carreira profissional do finalista Felipe foi construída na indústria automotiva, tendo atuado nos últimos (13) treze anos na Kromberg \& Schubert, uma empresa fabricante de chicotes elétricos. Apesar de seu início na área produtiva, não demorou muito para que o candidato fosse oferecido a oportunidade de transferência para a área de qualidade, onde vem trabalhando desde então.\\\\Durante sua trajetória, Felipe teve a oportunidade de atuar por um longo período na área de Inspeção,onde teve seu primeiro contato com fornecedores, participando de reuniões para buscar melhoriasnos processos em casos de não conformidade. No entanto, sua grande experiência com Qualidadede Fornecedores ocorreu nos últimos (05) cinco anos, quando o candidato assumiu o cargo de Analistade Qualidade. Desde então, Felipe vem trabalhando junto a um EQF em funções totalmente focadasna gestão de 57 fornecedores, sendo 48 destes manufatureiros (Injeção plástica, calhas, conectores,retentores, clipes, estamparia e cabos). Suas principais atividades estão na parte administrativa, sendo responsável pela elaboração dos PPAP, fornecendo suporte a todas as plantas da empresa na América Latina em casos de reclamações e oferecendo suporte à equipe de compras referente a certificações e análise de riscos dos fornecedores. Além disso, o candidato realiza visitas técnicas aos fornecedores para analisar os planos de ação e garantir melhorias, além de acompanhar o EQF (que atua como auditor líder) em auditorias VDA 6.3 e IATF 16949.\\\\Após um longo período em sua empresa atual, Felipe está em busca de mudanças e novos desafios na área. Ao ser abordado com a oportunidade na Sumidenso, o finalista se atraiu muito pela possível transição da cultura alemã, presente em seu emprego atual, para a cultura japonesa, na qual ele tem grande interesse devido à sua descendência. O candidato possui um perfil organizado e metódico, no entanto, valoriza a colaboração e não demonstra dificuldades em manter boas relações mesmo em situações complexas. O foco em detalhes e estabilidade são destaques em seu perfil, assim como sua preferência por ambientes estruturados e rotinas bem estabelecidas. 

\section{PACOTE DE REMUNERAÇÃO}
{\renewcommand{\arraystretch}{1.3} %<- modify value to suit your needs
\begin{tabular}{@{}p{0.02\textwidth}p{0.3\textwidth}p{0.53\textwidth}@{}}
  \resizebox{0.4em}{!}{$\blacksquare$} & Vínculo de Trabalho & CLT\\
  \resizebox{0.4em}{!}{$\blacksquare$} & Salário Fixo Bruto Mensal & 4.500,00\\
  \resizebox{0.4em}{!}{$\blacksquare$} & PLR & 2.300,00\\
  \resizebox{0.4em}{!}{$\blacksquare$} & Vale Refeição & 23,40\\
  \resizebox{0.4em}{!}{$\blacksquare$} & Vale Alimentação & 220,00\\
  \resizebox{0.4em}{!}{$\blacksquare$} & Assistência Médica & Intermédica, sem coparticipação e extensiva para dependentes.\\
  \resizebox{0.4em}{!}{$\blacksquare$} & Assistência Odontológica & N/A\\
  \resizebox{0.4em}{!}{$\blacksquare$} & Seguro de Vida & Itaú\\
  \resizebox{0.4em}{!}{$\blacksquare$} & Previdência Privada & N/A\\
  \resizebox{0.4em}{!}{$\blacksquare$} & Outros benefícios & Veículo coorporativo (Polo 2022/2023), partilhado entre colaboradores e mediante agendamento prévio para períodos longos e curtos.\\
\end{tabular}
}


Pretensão salarial: R\$ 6.800,00
\newline\newline
Fim do relatório.\\
São Paulo, \today.\\
Por: NOME DO FEITO POR\\
Revisão: NOME DA REVISÃO\\

\end{document}